\documentclass[
	12pt,
	a4paper,
	bibtotocnumbered
	]{scrreprt}		                                   

% Kodierung und Sprache/Rechtschreibung
\usepackage[utf8]{inputenc}

\usepackage[ngerman]{babel}
\usepackage[babel,german=quotes]{csquotes}

% Seitenränder
\usepackage[a4paper, left=2.5cm, right=3.0cm, top=2.5cm, bottom=2.0cm, includefoot]{geometry}

% Überschriften
\renewcommand*\chapterheadstartvskip{\vspace*{-\topskip}} % kein oberer Abstand

% Absatzlayout
\usepackage[parfill]{parskip}

% Maskierung von URLs und Dateipfaden
\usepackage[hyphens]{url}

%  Abbildungen 
\usepackage{graphicx}
\usepackage{caption}

%  Schriften 
\usepackage{mathptmx} % Times New Roman
\addtokomafont{disposition}{\rmfamily}
\usepackage[onehalfspacing]{setspace}

%  Misc 
\usepackage{blindtext}
\usepackage[printonlyused]{acronym}
\usepackage[linktoc=all, hidelinks]{hyperref}

% Inhaltsverzeichnis
\setcounter{tocdepth}{6}
\setcounter{secnumdepth}{3}

% Tabellen
\usepackage{tabularx}
\captionsetup[table]{belowskip=8pt}

%  Literatur 
\usepackage[backend=biber, hyperref=true, style=authoryear-icomp, maxnames=99]{biblatex}
\ExecuteBibliographyOptions{dashed=false}
\addbibresource{bibliography.bib}




\begin{document}



% ============================ FRONT MATTER
\newcommand{\thema}{Agile Entwicklung 2.0 - Ein Blick in sicherheitskritische Bereiche und DevOps}
\newcommand{\abgabedatum}{\today}
\newcommand{\zusammenfassung}{%

\emph{Agile Entwicklung} ist in der Praxis der Softwareentwicklung angekommen und nimmt weiter an Bedeutung zu.
Noch nutzen aber insbesondere die \emph{sicherheitskritischen Bereiche}, wie z. B. die Luft- und Raumfahrtindustrie teilweise klassische Softwareentwicklungsmodelle.
Auch der Betrieb von Software (\emph{DevOps}) könnte von den agilen Methoden profitieren.
Dadurch stellt sich die Frage, wie sich die Implementierung der agilen Entwicklung bzw. deren Methoden auf andere Bereiche anwenden lässt.

Als Vorgehensweise wird die Literatur-Review Methode verwendet. In den Grundlagen sollen klassische und agile Entwicklungsmodelle dargelegt werden. Des Weiteren wird eine Einführung in DevOps gegeben und klassische mit sicherheitskritischen Bereichen bezüglich verschiedener Faktoren verglichen. Dabei ist insbesondere für die agile Entwicklung von Bedeutung, in wie fern sich die Bereiche hinsichtlich dem Sicherheitsanspruch, Normen und verwendeten Entwicklungsmodellen unterscheiden.
Die Untersuchung der weitergehenden Implementierung der agilen Entwicklung und deren Methoden erstreckt sich über zwei Bereiche: Die \emph{horizontale} und \emph{vertikale Evolution}.

Die \emph{vertikale Evolution} betrifft die Anwendung agiler Entwicklungsmethoden auf die Vereinigung von Entwicklung und Betrieb von Software (DevOps).
Auf organisatorischer Ebene sind die Bereiche Entwicklung, Qualitätssicherung und Betrieb von Software oftmals vollständig getrennt. Es werden lediglich Erzeugnisse (Software, (Teil-)Systeme, etc.) zwischen den Abteilungen zur Bearbeitung ausgetauscht, was zu einem erhöhten Zeit- und Organisationsaufwand führt. DevOps versucht die Kommunikation und Zusammenarbeit zwischen diesen Bereichen zu verbessern und eine möglichst optimale Umgebung für die Entwicklung und Auslieferung von Software zu schaffen.
Hierbei wird erörtert, welche Voraussetzungen geschaffen werden müssen, um DevOps anzuwenden und welche Verbesserungen sich damit organisatorisch, wirtschaftlich und auf Ebene der Entwicklung erzielen lassen. Des Weiteren wird betrachtet, ob und wie  der DevOps Gedanke über die Grenzen der Entwicklung hinaus Anwendung finden kann.
Es wird ebenfalls betrachtet, wie DevOps in sicherheitskritischen Bereichen, teilweise oder auch vollständig, angewendet werden kann, worin Verbesserungspotentiale bestehen und wie diese vor dem Hintergrund von DevOps genutzt werden können.

Die Frage bezüglich \emph{horizontaler} Evolution beschäftigt sich mit der Implementierung von agiler Entwicklung in sicherheitskritischen Bereichen. 
Hierbei wird insbesondere geklärt, ob und unter welchen Voraussetzungen die agile Entwicklung in sicherheitskritischen Bereichen Einzug halten kann.
Diese Bereiche zeichnen sich durch andere Ansprüche bezüglich der Qualität, Korrektheit und der Time To Market von Software aus.
Hier gilt es auch zu beachten, das diese Bereiche teilweise durch Gesetze und Normen an gewisse Entwicklungsmodelle gebunden sein können.
Es soll auch dargelegt werden, welche Anpassungen für eine agile Entwicklung nötig sind und welche Vorteile sich daraus ergeben.

}

\pagenumbering{Roman}
\begin{singlespace}
\begin{titlepage}

\vspace*{-3.5cm}

\begin{flushleft}
\hspace*{-1cm} \includegraphics[width=15.7cm]{htwg-logo}
\end{flushleft}

\vspace{2.5cm}

\begin{center}
	\huge{
		\textbf{\sf \thema} \\[5cm]
	}
	\Large{
		\textbf{\sf Simon Kessler}} \\[6.5cm]
	\large{
		\textbf{\sf Konstanz, \abgabedatum} \\[2.3cm]
	}
	
	\Huge{
		\textbf{{\sf SEMINARARBEIT}}
	}
\end{center}

\end{titlepage}
\setcounter{page}{1}
\begin{center}
{\Large \textbf{Zusammenfassung (Abstract)}}
\end{center}

\bigskip

\begin{center}
	\begin{tabular}{p{2.8cm}p{10cm}}
		Thema: & \thema \\
		 & \\
		Abgabedatum: & \abgabedatum \\
		 & \\
	\end{tabular}
\end{center}

\bigskip

\noindent
\zusammenfassung
\end{singlespace}

\tableofcontents
\listoffigures
\chapter*{Abkürzungsverzeichnis} 

% Manuell alphabetisch sortieren!

\begin{acronym}

\acro{ABK}{Abkürzung}


\end{acronym}

\cleardoublepage
% ============================ ENDE FRONT MATTER


% ============================ HAUPTTEIL
\pagenumbering{arabic}


% ============================ CHAPTERS

\chapter{Einleitung} % 3 Seiten

Nachfolgend wird eine kurze Einführung in das Thema gegeben und die weitere Vorgehensweise erläutert.

\section{Gegenstand der Arbeit}

Diese Arbeit verwendet die Begriffe \emph{klassische} und \emph{sicherheitskritische} Bereiche zur Differenzierung zwischen Wirtschafts- und Unternehmensteilen die keinen erhöhten Sicherheitsbedarf haben und denen die explizit höhere Anforderungen bezüglich der Sicherheit haben (z. B. Kernkraftwerke, Luft- und Raumfahrtindustrie).
Die Verletzung dieser Ansprüche kann den Verlust streng vertraulicher Daten, Gefahr für Leib und Leben oder negative Auswirkungen für die Umwelt zur Folge haben.
Die Begriffe \emph{vertikale} und \emph{horizontale Evolution} (vgl. \autoref{fig:evolution}) werden im Kontext dieser Arbeit wie folgt definiert:
\emph{Vertikale Evolution} betrifft die Anwendung agiler Entwicklungsmethoden auf die Vereinigung von Entwicklung und Betrieb von Software (DevOps).
Dahingegen beschreibt \emph{horizontale Evolution} die Implementierung von agilen Vorgehensmodellen und Methoden in sicherheitskritischen Bereichen.

\begin{figure}
  \centering
  \includegraphics[width=\textwidth]{img/evolution.png}
  \caption{Vertikale und horizontale Evolution}
  \label{fig:evolution}
\end{figure}

Die agile Entwicklung ist nach diversen Studien und Umfragen bei den meisten Unternehmen im Alltag angekommen \parencite[vgl.][]{VersionOne:2015aa, HP:2015aa}. 
Damit hat sich der Trend fortgesetzt, von klassischen zu agilen Vorgehensmodellen über zu gehen \parencite[vgl.][]{Rodriguez:2012:SAL:2372251.2372275}.

Des Weiteren sind agile Vorgehensmodelle nicht nur auf die Entwicklung von Software beschränkt, sondern könnten auch auf den Betrieb von Software (DevOps) ausgeweitet werden.
Eine Fragestellung ist hierbei, welche Verbesserungen unter welchen Voraussetzungen durch die Anwendung von DevOps erreicht werden können und ob diese auch in sicherheitskritsichen Bereichen möglich sind.

In der gängigen Literatur wird meist nur auf die Anwendung von agilen Vorgehensmodellen in klassischen Bereichen eingegangen.
Insbesondere die sicherheitskritischen Bereiche, wie die Luft- und Raumfahrtindustrie, nutzen jedoch noch klassische Vorgehensmodelle.
Hier stellt sich die Frage, wie und ob diese Bereiche auch von den agilen Vorgehensmodellen profitieren können.

Die Relevanz der Fragestellung ergibt sich aus der Tatsache, dass gerade diese Bereiche oft unter hohen Lieferverzögerungen und Kostenüberschreitungen leiden.
Außerdem befindet sich beispielsweise die Raumfahrtindustrie in einem Wandel, da immer mehr private Unternehmen auf diesen Markt drängen, was einen erhöhten Wettbewerbsdruck verursacht.
Zwar finden sich zu den Themen agile Vorgehensmodelle und DevOps eine Vielzahl von Veröffentlichungen, jedoch nur sehr wenige beschäftigen sich mit der Anwendung und den Schwierigkeiten in sicherheitskritischen Bereichen.

\section{Zielsetzung}

Die vorliegende Arbeit soll untersuchen, wie und unter welchen Voraussetzungen sich DevOps einführen lässt, welche Vorteile sich dadurch für Entwicklung und Organisation ergeben und ob DevOps auch in sicherheitskritischen Bereichen Anwendung finden kann.
Außerdem soll geklärt werden, wie sich die Anwendung von agilen Vorgehensmodellen und deren Methoden auf sicherheitskritische Bereiche erweitern lässt.
Insbesondere soll untersucht werden, worin sich die klassischen und sicherheitskritischen Bereiche unterscheiden und welche Voraussetzungen geschaffen werden müssen, damit agile Methoden und DevOps eingeführt werden können.
Durch Betrachtung von Fallstudien, die die Einführung von DevOps und agilen Vorgehensmodellen in sicherheitskritischen Bereichen untersuchen, soll erarbeitet werden, welche Anpassungen an den Konzepten notwendig sind, um in diesen Bereichen sinnvoll eingesetzt werden zu können.
Abschließend sollen Schlussfolgerungen und Handlungsempfehlungen für sicherheitskritische Bereiche formuliert werden.


\section{Methodik}

Da die Arbeit bisherige Kenntnisse bezüglich sicherheitskritischen Bereichen, agilen Vorgehensmodellen und DevOps zusammentragen soll, bietet sich die Methode des \emph{Literatur-Reviews} an \parencite[vgl.][]{Fettke:2006aa}.
Die Arbeit entspricht einem natürlichsprachlichem Review mit dem Literaturfokus auf einer Kombination von Erfahrung (Anwendung in der Praxis) und der Theorie sowie empirischen Forschungsergebnissen.
Als Hauptziel sollen zentrale Aspekte der Arbeiten herausgearbeitet und Handlungsempfehlungen für die Praxis abgeleitet werden.
Hierbei beziehen die Autoren eine neutrale Haltung, wobei im Fazit eine Bewertung der Literaturergebnisse erfolgen soll.

Zur Suche wird auf die Quellen HTWG Bibliothek \parencite[][]{HTWGaa} zur Primärliteratursuche, Google Scholar und Google zur Sekundärliteratursuche zurückgegriffen.
Es werden die Stichwörter
\begin{itemize}
\item Agile
\item Agile Entwicklung / Agile Development
\item Vorgehensmodelle / Software Development Process
\item Sicherheitskritische Software / Security Critical Software
\item Luftfahrtindustrie / Air Industry
\item Raumfahrtindustrie / Space Industry
\item Softwaresicherheit / Software Security
\item Software Time To Market
\item Softwaresicherheit Anforderungen / Software Security Requirements
\item Softwarenormen / Software Standards
\item DevOps
\item Einführung DevOps / DevOps Adoption
\item DevOps Agile
\item DevOps Lean
\item DevOps Projektmanagement / DevOps Project Management
\item DevOps ITIL
\item DevOps Einsatz / DevOps Usage
\item DevOps Statistik / DevOps Statistics
\item DevOps in der Praxis / DevOps in Practice
\item DevOps Sicherheit / DevOps Security
\end{itemize}
zur Literatursuche benutzt.

Aus den Ergebnissen wird relevante Primärliteratur ausgewählt. 
Diese sollte sich mit den Kernthemen \enquote{sicherheitskritische Bereiche}, \enquote{DevOps}, \enquote{Softwarevorgehensmodelle} und \enquote{agile Vorgehensmodelle in sicherheitskritischen Bereichen} auseinandersetzen.
Bevorzugt wird Literatur ausgewählt, die in Konferenzsammlungen erschienen ist sowie die Anwendung in der Praxis untersucht.
Aus diesen Primärquellen wird mittels des Schneeballsystems (Zitate und Quellen rückwärts durchsucht) weitere relevante Literatur recherchiert.
Hier wird versucht, für den Hauptteil möglichst aktuelle Quellen zu verwenden.
Als Sekundärliteratur werden Studien und Arbeiten verwendet, die dem Rahmenwerk der Arbeit dienen und nicht die Hauptfragestellungen beantworten.
Alle Quellen werden in homogene Gruppen aufgeteilt und jeweils einige Vertreter daraus verwendet, so dass möglichst alle Aspekte der jeweiligen Thematik betrachtet werden.

Die Strukturierung der Arbeit verfolgt einen Thematisch orientierten Ansatz. 
Als Zielgruppe der Arbeit wird einerseits die Wissenschaft, als Anreiz für weitere Forschung in diesem Bereich, angesehen, als auch die Praxis, die hieraus Schlüsse über die Anwendbarkeit ziehen kann.

\section{Gang der Untersuchung}

Die Arbeit beginnt mit einer umfassenden Einführung in die Grundlagen, die klassische und agile Vorgehensmodelle vorstellt.
Des Weiteren wird der Begriff DevOps eingeführt und dessen geschichtliche Entstehung betrachtet.
Zur Verdeutlichung der Relevanz der Arbeit werden klassische und sicherheitskritische Bereiche bezüglich dem Anspruch an die Software verglichen.
Anschließend folgt die Betrachtung der vertikalen Evolution und ihrer Auswirkungen auf die Praxis.
Danach wird die horizontale Evolution analysiert, in dem zuerst die agilen Vorgehensmodelle in klassischen Bereichen und anschließend in sicherheitskritischen Bereichen untersucht werden.
Die Zusammenfassung der Arbeit und Handlungsempfehlungen für die Praxis sowie ein Ausblick auf weitere Forschungsarbeiten bilden den Schluss.




\chapter{Grundlagen}

\section{Entwicklungsmodelle}

\subsection{Klassisch}

\subsubsection{V-Modell}

\subsubsection{W-Modell}

\subsubsection{IBM RUP}

\subsubsection{Weitere ..}

\subsubsection{Probleme}

Probleme mit Kundenzufriedenheit, Kunde weiß nicht genau was er will (etc.),
fehlende Sichtbarkeit, unzureichendes Feedback, unzureichende Änderungsmöglichkeiten

\subsection{Agil}

\subsubsection{Scrum}

\subsubsection{Kanban}

\subsubsection{Xtreme-Programming}

\subsubsection{SEAP}

\subsubsection{Weitere spezielle für sicherheitskritische Branchen ..}
Weitere?

\section{DevOps}
Was ist DevOps?

\subsection{Geschichte und Entwicklung des Begriffs}
Woher stammt der Begriff DevOps und wie hat sich DevOps über die Zeit entwickelt?

\subsection{Ziele}
Was sind die Ziele von DevOps? Was ist daran so besonders/gut/neu?

\section{Klassische und sicherheitskritische Branchen im Vergleich}

\subsection{Time to Market}

\subsection{Sicherheitsanspruch}

\subsection{Normen}

Welche Normen existieren in kritischer SW? z. B. Regelungssoftware

\subsection{Risikomanagement}

Wie können Risiken verwaltet werden und welche Anforderungen bestehen diesbezüglich? (Normen etc. ?)

\section{Leistungskennzahlen}

Lines of Code, etc.
siehe auch Paper für Space Companies..

KPIs..
Bspw. Auswirkungen auf Projekt Management und soziale Faktoren

\chapter{Vertikale Evolution} % 13 Seiten
Dieses Kapitel betrachtet die vertikale Evolution der agilen Softwareentwicklung. Wie bereits erörtert, befinden sich Vorgehensweisen und Prozesse in einem stetigen Wandel. Agile Vorgehensweisen haben viele Probleme der Softwareentwicklung bereits gelöst, sind jedoch nur auf diesen Bereich beschränkt. DevOps hingegen umfasst einige weitere Ebenen, wodurch sich deutlich umfangreichere Möglichkeiten ergeben. Diese werden nachfolgend betrachtet.


\section{Abgrenzung zu Lean und Agile} % 1 Seite
Neben DevOps besteht noch eine Vielzahl an weiteren Prozessen und Vorgehensweisen, welche verwandt, aufbauend, oder vollkommen eigenständig sind. Zwei der am weitest verbreiteten sind agile Entwicklung und Lean. Beide werden oft im Zusammenhang mit DevOps genannt. Manchmal werden diese jedoch fälschlicherweise synonym verwendet. Daher erfolgt zuerst eine Abgrenzung zu diesen beiden Vorgehensweisen.

\subsection{Lean}
Die Lean Bewegung, geprägt von Eric Ries und dessen Buch \glqq The Lean Startup\grqq, verfolgt die Idee, der Prozessoptimierung und Konzentration auf Kernideen. Hierbei wird versucht, den Prozess von der Idee, über Ausarbeitung und Testing, hin zum fertigen Feature möglichst schlank und ressourcenschonend zu halten. Dies ist besonders im Start Up Umfeld sehr interessant, da dort finanzielle Mittle meist recht knapp bemessen sind und die Produkte noch nicht genau definiert sind. Mit Hilfe von Lean kann frühzeitig Kundenfeedback eingeholt werden, was die Entwicklung des Produkts und die Definition eines Zielmarkts deutlich vereinfacht.\\
Lean verfolgt den Ansatz, sich auf einige wenige Kernideen zu konzentrieren und nur Probleme zu lösen, die in Zusammenhang mit diesen stehen und deren Lösung sich lohnt. (NACHWEIS) \\
Da es sich bei Lean, manchmal auch Lean Management genannt, um Optimierung von Prozessen und Organisation handelt, findet sich diese Vorgehensweise auch außerhalb der IT wieder.

\subsection{Agile}
Agile Softwareentwicklung ist, wie bereits beschrieben, eine iterativ, inkrementelle Vorgehensweise. Hierbei werden lauffähige Prototypen und kurzen Zyklen erstellt und dem Kunde präsentiert. Dadurch besteht die Möglichkeit, Feedback des Kunden bereits frühzeitig einzuholen und darauf zu reagieren. Bedingt durch das iterative Vorgehen kann viel flexibler auf Änderungen der Spezifikation eingegangen werden und die Cost Of Change bleiben gering. Der Fokus liegt dabei auf Kollaboration mit dem Kunden und Bestreben, möglichst früh tatsächlichen Wert für den Kunden zu erzeugen. (NACHWEIS) Agile Vorgehensmodelle umfassen allerdings keine weiteren Bereiche wie beispielsweise Betrieb, oder Management, sondern sind nur auf die Entwicklung von Software beschränkt.

\subsection{DevOps}
DevOps baut auf die beiden oben genannten Vorgehensmodelle Lean und Agile auf, geht aber über deren jeweiligen Umfang hinaus. So beinhaltet DevOps beispielsweise das Lean Prinzip der Optimierung der Durchlaufzeiten, oder das Agile Prinzip der möglichst frühen Generierung von Wert für den Kunden. DevOps konzentriert sich aber nicht nur auf die technischen Aspekte, sondern auch auf die organisatorischen und kulturellen Ebenen. Hierbei stehen Kollaboration von Entwicklung und Betrieb und kultureller Wandel im Vordergrund.\\
Es muss jedoch beachtet werden, dass eine optimal funktionierende Lösung oftmals aus der Anwendung einer Kombination und nicht nur aus einem der drei Modelle besteht.


\section{Anwendung von DevOps in der Praxis} % 3 Seiten
Da es sich bei DevOps um ein Vorgehensmodell und nicht um ein fertig verfügbares Produkt handelt, kann die Anwendung in der Praxis, je nach Bedarf, viele unterschiedliche Formen annehmen. Nachfolgend werden beispielhaft die meist verbreiteten Möglichkeiten der Umsetzung dieser Praktiken vorgestellt.

\subsection{Organisatorisch}
Auf organisatorischer Ebene liegt der Fokus auf der Zusammenfassung der Abteilungen Entwicklung, Qualitätssicherung und Betrieb. Einer der Kernideen von DevOps ist die optimierte Kollaboration und Kommunikation in diesem Bereich. Um Wissen in einer Organisation möglichst transparent zu vorliegen zu haben, muss dieses durch Kommunikation und Kollaboration geteilt werden. Getrennte Abteilungen stellen hierbei jedoch meist sogenannte Silos dar, innerhalb derer Wissen unter Umständen geteilt wird, dieses aber normalerweise nicht nach Außen, beziehungsweise in andere Abteilungen gelangt. Durch die Anwendung von DevOps werden diese Silos abgeschafft und durch kollaborierende Teams mit geteiltem Wissen ersetzt. (NACHWEIS)
Eine Folge daraus ist, dass es keine Spezialisten klassischen Sinne mehr gibt. Teammitglieder müssen Fähigkeiten auf allen drei Gebieten, Entwicklung, Qualitätssicherung und Betrieb besitzen. Sie können aber immer noch bestimmte Fähigkeitsschwerpunkte besitzen. (NACHWEIS)

\subsection{Technisch}
DevOps ist lediglich ein Vorgehensmodell, oder eine Sammlung von Praktiken, in dessen Umfeld sich allerdings diverse Tools zur Unterstützung entwickelt haben. Einige dieser Tools stammen ursprünglich aus verwandten Bewegungen, oder wurden von anderen Bewegungen übernommen. Die Popularität einiger dieser Tools außerhalb von DevOps trägt sicherlich auch zur Verbreitung von DevOps bei.\\

\begin{figure}[ht]
  \centering
  \includegraphics[width=0.2\textwidth]{img/git_logo.png}
  \caption{Git Versionsverwaltung \parencite[][]{Git:2016}}
  \label{fig:scrummodell}
\end{figure}

Bei DevOps werden alle in der Produktion entstandenen Artefakte an einer zentralen Stelle verwaltet. Dies geschieht in den meisten Fällen mit einer sogenannten Versionsverwaltung. Der Vorteil solcher Tools liegt darin, dass kollaborativ und ortstransparent an Artefakten gearbeitet werden und jegliche Änderungen genau dokumentiert werden. Somit befindet sich ein damit verwaltetes System immer in einem genau definierten Zustand. Es besteht auch die Möglichkeit, Änderungen ohne großen Aufwand rückgängig zu machen, oder unterschiedliche Versionen parallel zu verwalten.\\
Ein sehr bekannter Vertreter dieser Gattung von Tools ist Git. Es wurde entwickelt von Linus Torvalds, dem Erfinder von Linux, und hat sich mittlerweile zu einem sehr weit verbreiteten Versionsverwaltungstool entwickelt. Eines der Kernfeature ist die dezentrale Speicherung von Entwicklerrepositories, die es ermöglichen ortstransparent und effizient im Team zu kollaborieren. (NACHWEIS)

\begin{figure}[ht]
  \centering
  \includegraphics[width=0.2\textwidth]{img/jenkins_logo.png}
  \caption{Jenkins Automatisierung \parencite[][]{Jenkins:2016}}
  \label{fig:scrummodell}
\end{figure}

Basierend auf dem Lean Prinzip der Optimierung, wird bei DevOps nicht nur die Organisation und Kollaboration optimiert, sondern auch auf technischer Ebene Entwicklung und Auslieferung. Als zentrales Element kommt hierbei oftmals eine Pipeline, in Form eines Continuous Integration Servers zum Einsatz. Eines der bekanntesten Open Source Projekte ist Jenkins. Dieses Projekt entstand aus dem Continuous Integration Server Projekt Hudson, nachdem dieses von Oracle aufgekauft wurde und viele Entwickler mit dessen Weiterentwicklung nicht mehr einverstanden waren. (NACHWEIS)\\
Ein Continuous Integration Server führt frei definierbare Aufgaben automatisiert aus und kann dabei mit unterschiedlichen Tools zusammenarbeiten. So kann er beispielsweise auf Änderungen in einer Versionsverwaltung reagieren und einen Bau- und Auslieferungsvorgang starten. Dabei können automatisierte Tests ausgeführt werden und das fertige Produkt auf verschiedene Systeme ausgeliefert werden. Die Einrichtung und Konfiguration solcher Continuous Delivery Server nimmt zu Beginn eines Projekts einige Zeit in Anspruch. Die zeitliche Ersparnis durch die vielen Möglichkeiten der Automatisierung gleichen diesen Aufwand aber wieder um ein Vielfaches aus. (NACHWEIS)\\
Continuous Integration Server bilden die Kerntechnologie bei DevOps, um eine schnelle Time To Market, schnelles Recovery nach Fehlerfällen, hohe Produktqualität und kurze Entwicklungszyklen zu erreichen.

\begin{figure}[ht]
  \centering
  \includegraphics[width=0.2\textwidth]{img/chef_logo.png}
  \caption{Chef Konfigurationsverwaltung \parencite[][]{Chef:2016}}
  \label{fig:scrummodell}
\end{figure}

\begin{figure}[ht]
  \centering
  \includegraphics[width=0.2\textwidth]{img/puppet_logo.png}
  \caption{Puppet Konfigurationsverwaltung \parencite[][]{Puppet:2016}}
  \label{fig:scrummodell}
\end{figure}

Ein weiteres Tool, welches Teil der Automatisierung ist und sehr gut mit einem Continuous Integration Server zusammen arbeitet ist eine Konfigurationsverwaltung. Diese Tools ermöglichen es, Konfigurationen zentral zu verwalten und automatisiert ablaufen zu lassen. Bei der manuellen Konfiguration von Servern und Systemen kommt es oft vor, dass sich Systeme in unterschiedlichen Zuständen befinden auf Grund unterschiedlicher Versionen, oder unterschiedlichen Herangehensweisen bei der Einrichtung. Dies bedeutet nicht nur bei der initialen Konfiguration, sondern auch bei allen darauf folgenden Updates und Upgrades einen erhöhten Zeitaufwand. Nicht standardisierte Systeme bedeuten auch im Fehlerfall einen erhöhten Aufwand bei der Suche und Beseitigung. Mit Hilfe von Konfigurationsverwaltungen erfolgt die Konfiguration aller System automatisiert, identisch und wiederholbar. Somit sinkt der Aufwand bei der initialen Konfiguration, bei Updates und Upgrades und im Falle eines Fehlers deutlich. Bekannteste Vertreter sind die Tools Chef und Puppet. Beide bieten eine sehr ähnlichen Grundfunktionalität, sind aber spezialisiert auf unterschiedliche Einsatzgebiete. (NACHWEIS)

\begin{figure}[ht]
  \centering
  \includegraphics[width=0.2\textwidth]{img/docker_logo.png}
  \caption{Docker Deployment \parencite[][]{Docker:2016}}
  \label{fig:scrummodell}
\end{figure}

Das Deployment, oder Auslieferung, bildet das abschließende Element einer automatisierten Pipeline. Hier werden alle Produktionsartefakte zu einem lauffähigen System zusammengefasst und auf die entsprechende Infrastruktur ausgeliefert. Beispielsweise die Installation einer Shop Applikation auf einem Webserver. In diesem Bereich gab es in den letzten Jahren auf Grund steigender Performanz bei Hardware und Software große Sprünge in der Entwicklung. Dies führt dazu, dass Infrastruktur, beziehungsweise Plattformen virtualisiert werden können. Damit vereinfacht sich die Entwicklung und die Auslieferung um ein Vielfaches. Ein solches Tool zur Auslieferung und Virtualisierung ist Docker.\\
Docker ermöglicht es, ein System vollkommen identisch beliebig oft auszuliefern. Somit kann es bei der Auslieferung von Systemen keine Probleme mehr in Form von unterschiedlichen Konfigurationen, oder Installationen geben. Auch die Entwicklung profitiert von Virtualisierung, da so Betriebsumgebungen im Kleinformat zur Entwicklung genutzt werden können und es keine Unterschiede mehr zwischen Entwicklung und Betrieb gibt. Zudem erhöht Virtualisierung die Sicherheit, da einzelne Systeme vollständig getrennt auf dem selben Server arbeiten können und Angreifer nicht zwischen diesen wechseln können. (NACHWEIS)

\subsection{Kulturell}
Ein oft vergessener, aber durchaus wichtiger Teil von DevOps ist der kulturelle Aspekt. Um diesen genauer betrachten zu können müssen zuerst die unterschiedlichen Kulturformen einer Organisation erörtert werden.
Nach Westrum \parencite[Vgl.][]{Westrum:1988} gibt es drei unterschiedliche Arten von Organisationen:

\begin{itemize}
\item Machtorientierte Organisationen
\item Bürokratische Organisationen
\item Generative und Leistungsorientierte Organisationen
\end{itemize}

In machtorientierte Organisationen gibt es nur sehr wenig Kooperation und Kollaboration. Zudem wird Scheitern als sehr negativ angesehen und meist nicht akzeptiert. Mitarbeiter arbeiten nur für den eigenen Vorteil und versuchen sich selbst möglichst unentbehrlich darzustellen. In einer solchen Art der Organisation gibt es nur sehr geringe, oder gar keine Innovation, da Mitarbeiter nicht gewillt sind, Risiken einzugehen und neue Richtungen einzuschlagen.\\
Bürokratische Organisationen zeichnen sich dadurch aus, dass jede Abteilung und jeder Mitarbeiter genau definierte Aufgaben und Ziele haben. Von diesen wird nicht abgewichen, auch wenn sich dadurch ein Vorteil für einen Einzelnen, oder das Team entstehen würde. In dieser Art der Organisation herrscht oftmals Angst vor Veränderungen und Neuerungen, dementsprechend findet auch hier nur sehr geringen, oder aber gar keine Innovation statt. Kollaboration und Kooperation werden nur angewendet, wenn dies genau vorgegeben ist.\\
In generativen und leistungsorientierten Organisationen findet hingegen sehr viel Kooperation und Kollaboration statt. Es herrscht ein fehlerverzeihendes Umfeld, in dem Mitarbeiter dazu ermutigt werden, Risiken einzugehen und eventuell zu scheitern, da alle Mitarbeiter dadurch lernen können. Mitarbeiter arbeiten an einem gemeinsamen Ziel und teilen die dabei entstehenden Risiken, anstatt sie einzelnen Abteilungen, oder Personen zuzuschreiben. In einer solchen Organisation existiert keine Art der \glqq über die Mauer werfen\grqq Mentalität, bei der Probleme \glqq über die Mauer\grqq an andere Abteilungen weitergegeben werden, ohne diese zu lösen, um eigenen Ressourcen zu schonen. Jeder Mitarbeiter hat eine Verantwortung für die Qualität, die Verfügbarkeit und die Sicherheit des Produkts. Ein solches Umfeld fördert Innovation und das Beschreiten neuer Wege.\\
DevOps strebt eine generative und leistungsorientierte Organisationskultur an, bei der Mitarbeiter, besonders die der Abteilungen Entwicklung, Qualitätssicherung und Betrieb in hohem Maße kollaborieren und kommunizieren. Durch die Zusammenfügung verschiedener Abteilungen wird das Risiko auf alle Mitarbeiter gleichermaßen verteilt und alle sind für die Qualität des Produkts verantwortlich. Nur so lässt sich ein innovatives und effizientes Umfeld aufbauen, weshalb die kulturelle Ebene für DevOps von so großer Wichtigkeit ist.

\newpage

\section{DevOps im Projektmanagement und ITIL} % 2 Seiten
Nachdem im vorangegangenen Abschnitt auf die praktische Umsetzung von DevOps eingegangen wurde, wird nun betrachtet, welche Anpassungen in Organisationen vorgenommen werden müssen, in denen andere Vorgehensmodelle eingesetzt werden und DevOps zum Einsatz kommen soll.

\subsection{Projektmanagement}



Management
kein ersatz, sondern Anpassungen
keine getrennten Abteilungen - ein Team
Auflösung der Silos - gemeinsame Verwaltung
Projektplanung: keine getrennten Phasen für Qualitätssicherung, etc.
Zeit einplanen für automatisierte Test
Demonstration: lauffähiger Prototyp -> demonstrieren, statt regelmäßig Bericht erstatten
Failure: mit dem Schlimmsten rechnen
schnelle Recovery (bringt DevOps im Idealfall mit)

\subsection{ITIL}
DevOps kein ITIL Ersatz
Verwendung in geeigneten Bereichen
- Vorteil durch Automatisierung und verbesserte Zusammenarbeit am größten sind

Vorgehensweise
- wichtigste ITIL Prozesse identifizieren in geeigneten Bereichen
- Review der Prozesse mit beteiligten Abteilungen (workshopartig)
- Identifizierung der Schwachstellen: meisten Kosten bei Fehlern?
- Wo kann Automatisierung und Kollaboration helfen?: nicht mehr Highlevel, sondern konkret (exakte Umsetzung)
- Priorisierung und Umsetzung


\section{Optimierungspotentiale und Verbreitung} % 1 Seiten

\subsection{Optimierungspotentiale}
Gartner
Einsatz verstärkt in Cloud und Mobile Branchen
IT-Führungskräfte befragt
Was hat Einführung von DevOps gebracht?
Schnellere Time to Market
Wiederverwendung / Automatisierung: Risikominimierung, Einsparung von Entwicklungsaufwand

\subsection{Verbreitung}
... siehe Bubbles


\section{Aktuelle Entwicklung von DevOps} % 3 Seiten
Bisher:
Nicht Teil des automatisierten Prozesses
Auf Entwicklung folgendes “Bottleneck”

\subsection{DevOps und Sicherheit}
Integration von Sicherheit in Pipeline: automatisierte Sicherheitstests

keine manuellen Änderungen am System (Überschreiben)
keine Updates -> nur Upgrades (neue Version)
Phoenix Upgrades / Blue-Green Upgrades

Immutable Infrastructure / Wegwerf Infrastruktur
Trennung von Daten und Infrastruktur (Rechte)

Auditierung und Compliance vereinfacht
- vereinfacht
- alles zentral und versioniert
- alle Änderungen protokolliert
- genau definierter Zustand
- Compliance Tests in Pipeline

Security as Software
- Vision
- komplette Sicherheit in Pipeline
- Automatisierung nicht einfach
- erfordert gewisse Abstraktion


\section{Fallstudie} % 3 Seiten
Nordstrom Fashion Retailer

\subsection{Einführung}
getrennte Abteilungen
lange Update Zyklen
langsame Reaktion auf Probleme
Homepage Upgrade (über Nacht, Ausfallzeiten)
Kassensystem Server Upgrade (Lange Dauer, Vor-Ort-Einsatz)

\subsection{Optimierungspotentiale durch DevOps}
in kleinem Bereich
Bezahlsysteme in den Läden
Virtualisierung der Bezahlsysteme
Virtualisierung -> einfaches Deployment (keine Arbeit vor Ort)
Windows Server 2003

Einsparung von Arbeitszeit und Aufwand
Kürzere Ausfallzeiten
schnellere Recovery

\subsection{Durchführung und Ergebnis}
Entwickler der Server, Datenbank, Website
Operations Leute, die vor Ort arbeiten

einige Wochen Arbeit
vollautomatisierte Erstellung in 4h statt 18h Arbeit vor Ort
Wiederholbar
Tests in Entwicklung
Entwicklung an original Systemen
Erfahrungen gesammelt, um Entwicklung des Herzstücks, der Homepage, zu automatisieren

\chapter{Horizontale Evolution} % 8 Seiten

Im Fokus dieses Kapitel steht die Einführung von agilen Vorgehensmodellen in sicherheitskritischen Branchen.
Damit die Unterschiede gegenüber den klassischen Branchen ersichtlich werden, sollen zuerst die Einführung in diesen Branchen thematisiert werden.
Dabei wird auf die Herausforderungen der einzelnen Branchen eingegangen.
Bei den sicherheitskritischen Branchen soll insbesondere auf die Luft- und Raumfahrtindustrie (NASA, ESA, etc.) eingegangen werden.

\section{Agile Vorgehensmodelle in klassischen Branchen} % 4 Seiten 

Nachfolgend wird auf die Auswirkungen von agilen Vorgehensmodellen, die Herausforderungen, Vor- und Nachteile in klassischen Branchen eingegangen.
Dazu werden mehrere Fallstudien analysiert.

\subsection{Kommunikation} % 1 Seite

Durch die Einführung von agilen Prozessen wird auch die Entscheidungsfindung in Unternehmen stark verändert.
Dabei wird von einem Führungsmodell mit starken Autorität zu einem verteilten, selbstverwaltenden Modell gewechselt.
Involviert sind die einzelnen Entwicklerteams sowie externe Stakeholder.
Es wird ein Kommunikationskonzept umgesetzt, welches den autoritäten Teil so weit wie möglich verringert.
Dies ist jedoch ein wesentlich komplexeres Modell, da die Entscheidungsfindung nun nicht mehr nur bei einer einzelnen Person, dem Manager, liegt.
Während in klassischen Vorgehensmodellen klare Vorgaben an die Teammitglieder gestellt werden, müssen diese in agilen Modellen auch Verantwortung für das Controlling ihrer Leistung und Kennzahlen übernehmen.
Dadurch wird auch implizit die Verantwortung einer Entscheidung jeweils an die Person verlagert, die das relevante Wissen bezüglich dieser Entscheidung besitzt.
Dies hat zur Folge, dass der klassische Projektmanager sich auf die reinen Verwaltungsaufgaben konzentrieren kann und sich nicht in fachliche Details einarbeiten muss.
Die Verlagerung der Entscheidung an näher am Problem liegende Stellen wird eine Verbesserung in der Lösung des Problems bezüglich Geschwindigkeit und Qualität erreicht.
Zur Umsetzung dieser selbstorganisierend Teams muss jedoch mehr umgesetzt werden, als nur flache Hierarchien und demokratische Prozesse zu implementieren.
Dabei sind folgende Bedingungen als effektiv eingestuft:
\begin{itemize}
\item Klare Ausrichtung der Teams
\item Eine leistungsfördernde Teamstruktur
\item Ein untersützender organisatorischer Kontext
\item Coaching durch einen Experten
\item Vorhandene und passende Ressourcen
\end{itemize}
\parencite[Vgl.][S. 863 f.]{Moe:2012aa}

Da kleine, selbstorganisierende Teams Personen nahe zueinander bringen, haben diese auch einen positiven Einfluss auf die generelle Kommunikation zwischen den Teammitgliedern.
Zusätzlich kann anfallende Dokumentation minimiert werden, die nur an den Schnittstellen zwischen Teams und Organisationseinheiten benötigt wurde.
Diese wird durch persönliche Kommunikation ersetzt, welche einen weiteren Effekt auf die Teams hat. 
Der persönliche Austausch wird gestärkt und damit auch das gegenseitige Lernen.
Dies wiederum erhöht das Verständnis aller Personen des Entwicklungsprozesses und verteilt das spezialisierte Wissen weniger Personen auf mehrere Personen (\emph{shared-knowledge}).
\parencite[Vgl.][S. 685]{Petersen:2010aa}

\subsection{Produktivität und Management} % 1 Seite

Die folgenden Erkenntnisse stützen sich auf die Fallstudien aus \parencite[][S. 418 ff.]{DeO.Melo:2013:ICS:2400747.2401010}.

Die Zusammensetzung der Teams hat einen starken Einfluss auf die Produktivität.
Dabei müssen insbesondere auch die Teamgröße, Persönlichkeiten und Fähigkeiten der einzelnen Personen beachtet werden.
Zwar hat die Kommunikation zwischen den Teams maßgeblichen Anteil an der Produktivität, jedoch stehen demgegenüber organisatorische Hindernisse (z. B. Unternehmensstrukturen, die die Kommunikation erschweren).

Wichtig für den Erfolg ist auch das Management der Personen ein sehr wichtiger Bestandteil bei der Implementierung von agilen Vorgehensmodellen.
Während in klassischen Vorgehensmodellen ein Fokus auf den Prozessen und formalen Richtlinien liegt, müssen bei den agilen Modellen die Personen und die Interaktion zwischen den Personen im Vordergrund stehen.
Dies stellt somit ganz andere Anforderungen an das Management.

Die Fluktuation im Team sollte möglichst gering gehalten werden, da diese sich negativ auf die Produktivität auswirkt. 
Um dem gegenüber zu treten, ist es ratsam verschiedene Konfliktbewältigungsmaßnahmen zu erproben.
Durch die selbstorganisierende Natur der Teams, sollte jeder Mitarbeiter diesen Zusammenhang und mögliche Gegenmaßnahmen verstehen und anwenden können.
Natürlich ist Fluktuation auch ein Zeichen für eine ungünstige Teamzusammensetzung, somit sollte bereits hier mehr Zeit investiert werden.

\subsection{Unternehmens- und Teamkultur} % 1/2 Seite

Agile Vorgehensmodelle stellen gewisse Anforderungen an das Unternehmen und die einzelnen Teams.
So nennen 42\% der Unternehmen einer Studie als Grund für das Scheitern von agilen Projekten inkompatible Unternehmenskulturen zu den agilen Grundwerten. 
Spezifischere Gründe sind eine starre Unternehmenskultur, die sich nicht verändern lässt sowie generell das Problem von unflexiblen Organisationsformen, die nicht gewillt sind, auf Veränderungen zu reagieren bzw. diese zuzulassen.
\parencite[Vgl.][S. 10]{VersionOne:2015aa}

Zusammenfassend muss erkannt werden, dass ein Unternehmen eine flexible Organisation besitzen oder erlauben sollte, die Änderungen im Entwicklungsprozess unterstützt.
Grundlage für solch eine Organisationsform ist eine Unternehmenskultur, die Änderungen nicht als Gefahr sondern als Chance erkennt und bereit ist, die Verantwortung in der Hierarchie nach unten zu propagieren.

\subsection{Vorteile} % 1/2 Seite

Ein Hauptteil der Vorteile lässt sich unter dem Sammelbegriff \emph{Kommunikation} zusammenfassen.
Es wird eine verbesserte Kommunikation zwischen allen beteiligten Stakeholdern genannt.
Als Grund lassen sich die kurzen Iterationszyklen identifizieren. 
Des Weiteren trägt dazu bei, dass der Kunde möglichst in der Nähe der Entwickler sein sollte, um Probleme und Unklarheiten klären zu können.
Von den Entwicklern selbst wird der agile Prozess mehrheitlich positiv aufgenommen, da diese sich wertgeschätzt fühlen.
\parencite[Vgl.][S. 658]{Petersen:2010aa}

Auch das Vertrauen zwischen Kunde und Entwicklern wird durch den agilen Prozess verbessert.
Aus Kundensicht wird eine verbesserte Verteilung des Wissens unter den Entwicklern empfunden.
So wird es in einem agilen Prozess als wahrscheinlicher angesehen, von einem beliebigen Teammitglied spezifische Informationen zum Projekt zu erhalten als in einem klassischen Vorgehensmodell.
\parencite[Vgl.][S. 498 f.]{Bomarius:2005aa}

Aus Projektmanagementsicht verbessert sich die Projektsichtbarkeit, d.h. es ist für das Management einfacher den aktuellen Stand des Projekts zu beurteilen.
Weitere Vorteile sind das bessere Management von sich ändernden Prioritäten, erhöhte Produktivität, schnellere Time to Market und eine höhere Zufriedenheit des Teams.
\parencite[Vgl.][]{VersionOne:2015aa}

\subsection{Nachteile} % 1/2 Seite

Zwar lässt die Prämisse, dass kleine, selbstorganisierende Teams eingesetzt werden darauf schließen, dass sich das Vorgehensmodell gut skalieren lässt. 
Jedoch konnte diese Vermutung in der Praxis nicht bestätigt werden. 
Es entsteht ein erhöhter Kommunikationsaufwand, wenn mehrere verteilte Teams auf das selbe Ziel hinarbeiten.
Dadurch sind viele verschiedene Personen in der Kommunikation involviert und dementsprechend steigt die Komplexität.

Des Weiteren wird wenig Wert auf die Architektur gelegt, da es einem umfassenden Architekturüberblick fehlt. 
Dies führt zu einer tendenziell schlechteren Architektur.
Testen und kontinuierliche Integration sind nur unter hohem Aufwand realisierbar.
Denn durch die kurzen Entwicklungszyklen steigt auch der Wartunsaufwand, da Kunden nun eine Vielzahl von Versionen zu einem Zeitpunkt einsetzen.
Dies wiederum erhöht die Komplexität bei der Versionsverwaltung.
\parencite[Vgl.][S. 1486]{Petersen20091479}

Ein Problem ist auch die Abwägung zwischen kurzfristiger und langfristiger Qualität am Ende eines Sprints.
Einerseits steht der Kundennutzen am Ende eines Sprints im Vordergrund, andererseits sollte das Produkt jedoch auch langfristig wart- und erweiterbar bleiben.
Zwar wird die Einführung von agilen Methoden als wichtige strategische Entscheidung betrachtet, jedoch müssen dazu auch die beteiligten Personen auf operationeller Ebene dieses Vorgehen unterstützen.
\parencite[Vgl.][S. 863 f.]{Moe:2012aa}



\newpage
\section{Agile Vorgehensmodelle in sicherheitskritischen Branchen} % 4 Seiten

Dieser Abschnitt soll die Frage klären, wie agile Vorgehensmodelle u.a. in der Luft- und Raumfahrtindustrie etabliert werden und welche Vor- und Nachteile dadurch entstehen können.

\subsection{Anpassungen der Vorgehensmodelle} 

Nachfolgend wird gezeigt, wie agile Vorgehensmodelle angepasst werden können, um den erhöhten Anforderungen der sicherheitskritischen Branchen Rechnung zu tragen.

\subsubsection{SEAP} % 1 Seite

Mit einem Fokus auf kosteneffiziente Prozesse und Tools integriert SEAP (vgl. \autoref{sec:seap}) sicherheitsrelevante Überlegeungen in agile Vorgehensmodelle.
\parencite[][]{Baca:2015aa} untersucht die Einführung von SEAP bei Ericsson AB in einem Geldtransfersystem und evaluiert die Ergebnisse.
Ericsson benutzt ein martkgetriebenes Entwicklungsmodell, bei dem die Requriements von einer Vielzahl an Kunden gesammelt werden.
Des Weiteren bestehen die Anforderungen an das Geldtransfersystem in einem sehr hohen Sicherheitsanspruch und einer länderspezifischen Anpassung.

Nach der Einführung von SEAP konnten bei Ericsson mehrere Verbesserungen gemessen wurden.
So verbesserte sich die Anzahl der erkannten, schweren Risiken deutlich. 
Das heißt, die Grundvoraussetzung, das Erkennen von Risiken, wurde alleine durch SEAP signifikant verbessert.
Darauf aufbauend wurde analysiert, wie viele der erkannten Risiken verhindert wurden.
Die Anzahl der beseitigten Risiken wurde um 54 \% erhöht.
Außerdem konnte die Anzahl der Risiken, die auf die nächste Version verschoben wurden, konnte um 27 \% verringert werden.
Zusätzlich wurde eine Reduktion der Anzahl der nicht beseitigen Risiken um 29 \% erreicht.
Trotz dieser Verbesserungen konnte eine Verringerung der Mannstunden pro identifiziertem Risiko um 45 \% auf 1,48 h erreicht werden.
Außerdem wurde der Prozess der Identifizierung der Risiken verbessert werden, so dass nur noch ca. 1,7 h anstatt 21,6 h nötig sind.
\parencite[Vgl.][S. 17 ff.]{Baca:2015aa}

\subsubsection{Requirements Engineering} % 1/2 Seite

\parencite[][]{peeters2005agile} stellt einige Erweiterungen für das Requirements Engineering in agilen Prozessen vor.
So werden die User Stories um sogenannte \emph{Abuser Stories} ergänzt.
Eine Abuser Story ähnelt formal einer User Story, beschreibt jedoch wie ein Angreifer das System missbrauchen kann, um an Daten o.ä. zu kommen.
Gegenüber einer User Story wird eine Abuser Story nicht über den Wertbeitrag priorisiert, sondern über das potentielle Risiko und dem Schaden durch den Angriff.
Dabei wird einerseits der potentielle Schaden, andererseits die Eintrittswahrscheinlichkeit zur Berechnung herangezogen.
Eine Abuser Story sollte, bei Bezug zu einer User Story, Kosten in gleicher Höhe wie der Wertbeitrag der User Story besitzen.
Dadurch kann die Abuser Story als Umkehrfunktion betrachtet werden, die den Wertbeitrag der User Story wieder vernichtet, wenn das Szenario eintritt.

Die Aufwandsschätzung einer Abuser Story bezieht sich dabei auf den Aufwand, der nötig ist, um das System gegen den Angriff zu schützen.
Eine \emph{Widerlegung} ist das Gegenstück zu einem Test. 
Sie zeigt, dass durch die getroffenen Gegenmaßnahmen der Angriff nicht mehr möglich ist bzw. die Wahrscheinlichkeit auf ein erträgliches Maß reduziert wurde.

Diese Ergänzung des agilen Modells, stellt eine einfache Art dar, die Requirements Engineering Phase durch Sicherheitsbetrachtungen zu ergänzen, ohne die grundlegenden agilen Abläufe maßgeblich zu verändern.

\subsubsection{ECSS} % 1 Seite 

\subsubsection{Extreme Programming} % 1 Seite

\subsubsection{Abuser Stories} % 1/2 Seite



\subsection{Kompatibilität mit Normen}






\chapter{Schluss}

\section{Zusammenfassung}

Im Schlussteil sollte man daher die in der Einleitung formulierten Leitfragen noch einmal aufgreifen und auf deren Beantwortung im Hauptteil der Hausarbeit eingehen: Wurden alle Leitfragen aus der Einleitung beantwortet?
Gleichzeitig bietet die Beantwortung der Leitfragen auch eine Möglichkeit, die Ergebnisse der Hausarbeit "auf den Punkt" bringen und eine Zusammenfassung zu formulieren.

\section{Fazit \& Ausblick}

Zudem sollte der Schlussteil immer eine Interpretation und Bewertung der Forschungsergebnisse enthalten: Was leisten die Ergebnisse? Haben sich durch Ihre Hausarbeit (Forschungs-)Fragen ergeben, die noch geklärt werden müssen (sog. Ausblick)?



% ============================ ENDE HAUPTTEIL

% ============================ LITERATURVERZEICHNIS

%\begin{singlespace}
\printbibliography
%\end{singlespace}

% ============================ ENDE LITERATURVERZEICHNIS



\end{document}