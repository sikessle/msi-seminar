\chapter{Einleitung} % 3 Seiten

Nachfolgend wird eine kurze Einführung in das Thema gegeben und die weitere Vorgehensweise erläutert.

\section{Gegenstand der Arbeit}

Diese Arbeit verwendet die Begriffe \emph{klassische} und \emph{sicherheitskritische} Bereiche zur Differenzierung zwischen Wirtschafts- und Unternehmensteilen die keinen erhöhten Sicherheitsbedarf haben und denen die explizit höhere Anforderungen bezüglich der Sicherheit haben (Kernkraftwerke, Luft- und Raumfahrtindustrie, etc.).

Die agile Entwicklung ist nach diversen Studien und Umfragen \parencite[vgl.][]{VersionOne:2015aa, HP:2015aa} bei den meisten Unternehmen im Alltag angekommen.
Damit hat sich der Trend fortgesetzt, von klassischen zu agilen Vorgehensmodellen über zu gehen. \parencite[Vgl.][]{Rodriguez:2012:SAL:2372251.2372275}
In der gängigen Literatur wird meist nur auf die Anwendung von agilen Methoden in klassischen Bereichen eingegangen.
Insbesondere die sicherheitskritischen Bereiche, wie die Luft- und Raumfahrtindustrie, nutzen jedoch noch klassische Vorgehensmodelle.
Dies ist oft bedingt durch nationale Gesetze und Normen.
Hier stellt sich die Frage, wie und ob diese Bereiche auch von den agilen Vorgehensmodellen profitieren könnten.

Des Weiteren sind agile Vorgehensmodelle nicht nur auf die Entwicklung von Software beschränkt, sondern könnten auch auf den Betrieb von Software (DevOps) ausgeweitet werden.
Auch hier kann unter Umständen in sicherheitskritischen Bereichen eine Verbesserung verschiedenster Faktoren durch die Einführung von DevOps erreicht werden.
Jedoch müssen auch hier die besonderen Gegebenheiten dieser Bereiche beachtet werden.

Die Relevanz der Fragestellung ergibt sich aus der Tatsache, dass gerade diese Bereiche oft unter hohen Lieferverzögerungen und Kostenüberschreitungen leiden.
Zwar finden sich zu den Themen agile Vorgehensmodelle und DevOps eine Vielzahl von Veröffentlichungen, jedoch nur sehr wenige beschäftigen sich mit der Anwendung und den Schwierigkeiten in sicherheitskritischen Bereichen.

\section{Zielsetzung}

Die vorliegende Arbeit soll untersuchen, wie sich die Implementierung der agilen Entwicklung und deren Methoden auf sicherheitskritische Bereiche erweitern lässt.
Außerdem soll geklärt werden, wie DevOps durch agile Methoden profitieren und in sicherheitskritischen Bereichen angewendet werden kann.
Insbesondere soll untersucht werden, worin sich die klassischen und sicherheitskritischen Bereiche unterscheiden und welche Voraussetzungen geschaffen werden müssen, damit agile Methoden und DevOps eingeführt werden können.
Zusätzlich werden die Auswirkungen dieser Veränderung auf die Bereiche Organisation, Entwicklungsprozess und Qualität untersucht.
Abschließend sollen Schlussfolgerungen und Handlungsempfehlungen für sicherheitskritische Bereiche und insbesondere die Luft- und Raumfahrtindustrie formuliert werden.


\section{Methodik}

Da die Arbeit bisherige Kenntnisse bezüglich sicherheitskritischen Bereichen, agilen Vorgehensmodellen und DevOps zusammentragen soll, bietet sich die Methode des \emph{Literatur-Reviews} \parencite[vgl.][]{Fettke:2006aa} an.
Somit kann der aktuelle Stand der Forschung erfasst und interpretiert werden.
Die grundlegenden Themen agile Entwicklung und DevOps finden sich in vielen Veröffentlichungen wieder, somit soll hierbei eine Komprimierung auf die wesentlichen, allgemein anerkannten Aspekte stattfinden.

\section{Gang der Untersuchung}

Die Arbeit beginnt mit einer umfassenden Einführung in die Grundlagen.
Dabei sollen klassische und agile Vorgehensmodelle vorgestellt werden.
Des Weiteren wird der Begriff DevOps eingeführt.
Zur Erreichung der Ziele dieser Arbeit, werden \emph{klassische} und \emph{sicherheitskritische} Bereiche bezüglich objektiver Faktoren verglichen.
Abschließend bieten die Grundlagen eine Erläuterung von messbaren Leistungskennzahlen, die zur Erfolgsmessung der Vorgehensmodelle und DevOps in den Bereichen eingesetzt werden können.

Zur Suche wird auf die Quellen HTWG Bibliothek \parencite[vgl.][]{HTWG2015aa}, Google Scholar \parencite[vgl.][]{Google2015aa} und Google \parencite[vgl.][]{Google2015ab} zurückgegriffen.
Unter anderem werden die Stichwörter (und deren englische Entsprechung)
\begin{itemize}
\item Agile
\item Agile Entwicklung
\item Vorgehensmodelle
\item Sicherheitskritische Branchen und Bereiche
\item Luft- und Raumfahrtindustrie
\item Softwaresicherheit
\item Software Kennzahlen
\item Software Time to Market
\item Softwaresicherheit Anforderungen
\item Softwarenormen
\item DevOps Stichwörter HIER ERGÄNZEN
\end{itemize}
zur Literatursuche benutzt.
Die gefundene Literatur dient als weitere Quelle für relevante Literatur, indem Zitate und Quellenverweise vorwärts (wer hat zitiert?) und rückwärts (wer wurde zitiert?) durchsucht werden.
