\chapter{Einleitung}

\section{Gegenstand der Arbeit}

In der Einleitung/Problemstellung ist zunächst das Thema der Hausarbeit vorzustellen. Dabei ist das Thema in einen größeren Zusammenhang (z. B. inhaltlich oder zeitlich) einzuordnen. Warum ist es interessant und forschungswürdig, sich mit genau dieser Fragestellung auseinanderzusetzen?

\section{Zielsetzung}

Im Anschluss ist der Leser über die Zielsetzung der Hausarbeit zu informieren. In der Zielsetzung (Umfang: max. 4-5 Sätze) ist prägnant, trennscharf, genau und weitgehend interpretationsfrei zu beschreiben, was im Rahmen Ihrer Arbeit untersucht wird (selbstverständlich bezogen auf die Themenstellung).
Auf welche wissenschaftliche Fragestellung soll Ihre Arbeit eine Antwort geben? Welche Position/These in der Fachliteratur soll kritisch hinterfragt werden?

Insgesamt sollen anhand der Zielsetzung eine oder mehrere Leitfragen aufgestellt werden, die der Hausarbeit zugrunde liegen und die im Hauptteil beantwortet werden.
Darüber hinaus müssen Sie im Rahmen der Zielsetzung der Arbeit eine sinnvolle Eingrenzung der Untersuchung vornehmen, sofern dies erforderlich ist

\section{Methodik}

??
Literatur Review?

\section{Gang der Untersuchung}

Nach der Formulierung der Zielsetzung der Hausarbeit ist es sinnvoll, dem Leser einen groben Überblick über den Aufbau der Argumentation zu geben: Welche Schritte werden durchgeführt, um das Ziel zu erreichen – und warum wird genau so vorgegangen (Begründung der Vorgehensweise)? Die Beschreibung des Ganges der Untersuchung ist somit "mehr" als nur das "bloße Wiedergeben" der Gliederung. Durch diese Erläuterungen weiß der Leser bereits am Anfang, was ihn im Verlauf der Hausarbeit erwartet, er kann sich darauf einstellen und daran orientieren.

Wichtig: Nennen der untersuchten Branchen (grob, Details später im jeweiligen Kapitel).


%%%%%
Ein Beispielsatz. \parencite[Vgl.][S. 20]{Thompson:1984:RTT:358198.358210}

Neuer Absatz sdf

- Gegenstand der Arbeit
- Problemstellung und Ziel der Arbeit 
- Methodik
- Gang der Untersuchung

Als Arbeitstitel habe ich euren Vorschlag aufgenommen: „Agile Entwicklung 2.0 - what's next?“. Ein Blick in die Luft- und Raumfahrtindustrie.

Wichtig wäre mir noch Folgendes: 
- Das Thema agile Entwicklungsmethoden steht - wie ihr richtigerweise angemerkt habt - auch den anderen als Wahlmöglichkeit zur Auswahl. Hier sollten wir dann eine Abstimmung hinsichtlich der Präsentation erreichen (Für die Ausarbeitung ist dies nicht relevant). Ich stelle mir das so vor, dass das Thema aus dem Wahlkatalog auf die Grundlagen der agilen Methode eingeht und ihr darauf aufbauend eure Folgethematik beschreibt. Aber das klären wir einfach bis...
- Beachtet den Managementfokus, den das Seminar legt. Angaben z.B. zur Messung von Kosten- und Performance-Vorteilen wären interessant. Welche Erfahrungen bestehen dazu?...