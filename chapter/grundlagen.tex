\chapter{Grundlagen} % 15 Seiten

Die Vermittlung der Grundlagen dient als Basis für die weitere Arbeit und beginnt mit  einer Erläuterung der verschiedenen Vorgehensmodelle.
Darauf folgt eine Einführung in DevOps bevor das Kapitel mit einem Vergleich der Branchen und einer kurzen Darstellung von wichtigen Leistungskennzahlen abschließt.

\section{Vorgehensmodelle} % 6 Seiten

Diese Arbeit kategorisiert die Vorgehensmodelle zur Entwicklung von Software in zwei Kategorien: \emph{klassische} und \emph{agile} Vorgehensmodelle.
Unter \emph{klassisch} werden alle Vorgehensmodelle, die nicht agil und somit plangetrieben sind zusammengefasst.
Im Fokus dieses Abschnitts steht die Erläuterung relevanter Vertreter dieser Kategorien.

FOKUS AUF MANAGEMENT

\subsection{Klassisch} % 3 Seiten

Nachfolgend sollen einige wichtige klassische iterative als auch nicht-iterative Vorgehensmodelle kurz vorgestellt werden.

\subsection{Wasserfallmodell}

Das Wasserfallmodell wird konzeptionell das erste mal in \parencite[][]{Royce:1987aa} beschrieben.

Name "Wasserfallmodell" erstmals erwähnt. \parencite[Vgl.][S. XY]{Benington:1983aa}

Finde heutzutage kaum noch Anwendung.  \parencite[Vgl.][S. 48]{Schatten:2010aa}

\subsection{Spiralmodell}

\subsubsection{V-Modell}

\subsubsection{V-Modell XT}

\subsubsection{W-Modell}

\subsubsection{IBM Rational Unified Process (RUP)}


\subsection{Agil} % 3 Seiten

MANIFESTO!

\parencite[Vgl.][S. XY]{6979143}  Da gibt es ein Bild. Vergleich Agile/Normal mit Management etc.

\subsubsection{Scrum}

\subsubsection{Kanban}

\subsubsection{Extreme Programming}

\subsubsection{SEAP}

\section{DevOps} % 3 Seiten
Was ist DevOps?

\subsection{Geschichte und Entwicklung des Begriffs}
Woher stammt der Begriff DevOps und wie hat sich DevOps über die Zeit entwickelt?

\subsection{Ziele}
Was sind die Ziele von DevOps? Was ist daran so besonders/gut/neu?

\section{Klassische und sicherheitskritische Branchen im Vergleich} % 4 Seiten

Definition sicherheitskritisch (Leben, etc.)

\subsection{Time to Market}

\subsection{Sicherheitsanspruch}

\subsection{Normen}

Welche Normen existieren in kritischer SW? z. B. Regelungssoftware

\subsection{Risikomanagement}

Wie können Risiken verwaltet werden und welche Anforderungen bestehen diesbezüglich? (Normen etc. ?)

\subsection{Vertragliche Vereinbarungen}

FESTPREIS!

\section{Leistungskennzahlen} % 2 Seiten

Lines of Code, etc.
siehe auch Paper für Space Companies..

KPIs..
Bspw. Auswirkungen auf Projekt Management und soziale Faktoren
