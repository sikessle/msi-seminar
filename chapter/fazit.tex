\chapter{Schluss} % 2 Seiten

Abschließend erfolgt eine kurze Zusammenfassung und ein Fazit sowie ein Ausblick, welche weiteren Arbeiten in diesem Bereich angestrebt werden könnten.

\section{Zusammenfassung}

\begin{comment}
++++++++++++
Im Schlussteil sollte man daher die in der Einleitung formulierten Leitfragen noch einmal aufgreifen und auf deren Beantwortung im Hauptteil der Hausarbeit eingehen: Wurden alle Leitfragen aus der Einleitung beantwortet?
Gleichzeitig bietet die Beantwortung der Leitfragen auch eine Möglichkeit, die Ergebnisse der Hausarbeit "auf den Punkt" bringen und eine Zusammenfassung zu formulieren.

FRAGEN AUS EINLEITUNG:
"Die vorliegende Arbeit soll untersuchen, wie und unter welchen Voraussetzungen sich DevOps einführen lässt und ob DevOps auch in sicherheitskritischen Bereichen Anwendung finden kann.
Außerdem soll geklärt werden, wie sich die Anwendung von agilen Vorgehensmodellen und deren Methoden auf sicherheitskritische Bereiche erweitern lässt.
Insbesondere soll untersucht werden, worin sich die klassischen und sicherheitskritischen Bereiche unterscheiden und welche Voraussetzungen geschaffen werden müssen, damit agile Methoden und DevOps eingeführt werden können.
Durch Betrachtung von Fallstudien, die die Einführung von DevOps und agilen Vorgehensmodellen in sicherheitskritischen Bereichen untersuchen, soll erarbeitet werden, welche Anpassungen an den Konzepten notwendig sind, um in diesen Bereichen sinnvoll eingesetzt werden zu können.

=> das beantworten..
++++++++++++
\end{comment}


% DEVOPS HIER EINFÜGEN

Die Betrachtung der horizontalen Evolution beschäftigt sich mit der Frage, ob und wie agile Vorgehensmodelle und deren Methoden in sicherheitskritischen Bereichen angewendet werden können.

Damit die Fragestellung einen validen Ausgangspunkt hat, werden in den Grundlagen die klassischen und sicherheitskritischen Bereiche miteinander verglichen.
Hier wird deutlich, dass sicherheitskritische Bereiche einen stark erhöhten Sicherheitsanspruch an die Software haben, andere (strengere) Normen Anwendung finden, bei den Entwicklungspraktiken auf stark formalisierte Vorgehensmodelle gesetzt wird und die Time To Market im Vergleich unwichtiger ist.

Zur Klärung der horizontalen Evolution wird zuerst betrachtet, wie sich agile Vorgehensmodelle in klassischen Bereichen auswirken.
Dabei wird deutlich, dass die generelle Kommunikation zwar verbessert wird, dies jedoch auch eine deutlich höhere Komplexität mit sich bringt, insbesondere in verteilten Teams.
Des Weiteren steigt die Produktivität, sofern die Teams wenig Fluktuation aufweisen und das Management es versteht, Menschen und deren Interaktionen gut zu managen.
Dies führt zur nächsten Beobachtung, dass die Unternehmens- und Teamkultur einen maßgeblichen Anteil daran hat, ob agile Vorgehensmodelle erfolgreich eingesetzt werden können.
Weitere Vorteile sind eine verbesserte Kommunikation zwischen Team und Kunde, eine erhöhte Projektsichtbarkeit, schnellere Reaktion auf Änderungen, höhere Zufriedenheit der Teams sowie eine schnellere Time To Market.

Die Untersuchung, ob und wie sich agile Vorgehensmodelle in sicherheitskritischen Bereichen einführen lassen, wird anhand mehrerer Kernpunkte durchgeführt.
Es wird deutlich, dass sich Scrum in der Raumfahrtindustrie grundsätzlich einsetzen lässt, wenn Scrum in Kombination mit einem klassischen Vorgehensmodell in Form eines hybriden Vorgehensmodells eingesetzt und angepasst wird.
Dabei sollte Scrum für neuartige Prozesse, die Wissen generieren und klassische Vorgehensmodelle für Routineprozesse verwendet werden.
Einige Scrum-Bestandteile wie sich ändernde Requirements, minimale Dokumentation und Refactoring erweisen sich als ungünstig und müssen verworfen werden.
Dahingegen sind testgetriebene Entwicklung, Pair-Programming und häufige, kurze Meetings sehr gut geeignet.
Extreme Programming muss noch stärker angepasst werden, in dem die Spezifikation von Sicherheitsanforderungen, der Aufbau eines Zusicherungsniveaus und dessen Verifikation mittels Tests durch Prozesse besser unterstützt werden.
SEAP stellt einen neuartigen Ansatz dar, wie agile Vorgehensmodelle durch sicherheitsrelevante Betrachtungen und Prozesse ergänzt werden können, ohne die Agilität zu verlieren.
Dabei werden sicherheitsbezogene Rollen eingeführt die in engem Austausch mit dem Entwicklerteam stehen.
Des Weiteren wird eine iterative Risikoanalyse integriert.
Die Einführung von SEAP hatte bei Ericcson deutliche Verbesserungen in qualitativer sowie quantitativer Form zur Wirkung.
So konnte eine Steigerung der Erkennungsrate von Risiken sowie deren Beseitigung erreicht werden. 
Jedoch stiege zugleich auch die Personalkosten an.
Eine weitere agile Methodik ist die Ergänzung von User Storys um Abuser Storys, die potentielle Angriffe und deren Schaden einer User Story zuordnen.
Sie bilden eine einfache Möglichkeit sicherheitsrelevante Betrachtungen in agile Vorgehensmodelle zu integrieren.
Dass die Normen in sicherheitskritischen Bereichen grundsätzlich auch mit agilen Vorgehensmodellen erfüllbar sind, wird durch Analyse der relevanten Normen klar.
Für die Norm der ECSS wird eine beispielhafte Implementierung bei der Swedish Space Company dargelegt.

\section{Fazit \& Ausblick}

\begin{comment}
++++++++++++
Zudem sollte der Schlussteil immer eine Interpretation und Bewertung der Forschungsergebnisse enthalten: Was leisten die Ergebnisse? Haben sich durch Ihre Hausarbeit (Forschungs-)Fragen ergeben, die noch geklärt werden müssen (sog. Ausblick)?

Stand der Dinge in Sachen DevOps? Kann es sich durchsetzen und halten?



Schlussfolgerungen.

Handlungsempfehlungen für die Praxis!
++++++++++++
\end{comment}


% DEVOPS HIER EINFÜGEN

Die untersuchte Literatur legt den Schluss nahe, dass agile Vorgehensmodelle und Methoden grundsätzlich in sicherheitskritischen Bereichen eingeführt werden können.
Dazu sind jedoch noch Anpassungen in Bezug auf die Dokumentation an den Modellen notwendig, was wiederum zum Teil im Widerspruch zu den agilen Grundwerten steht.
Diese Anpassungen sind kritisch zu sehen und sollten in der Praxis sehr behutsam eingeführt werden, da ansonsten eine Formalisierung der agilen Methoden zu befürchten ist, was kontraproduktiv wäre.
Denn vor dem Hintergrund der zunehmenden Privatisierung der Raumfahrt sind in diesem Bereich dringend agile Vorgehensmodelle nötig, um im Umfeld des erhöhten Wettbewerbsdruck durch eine effizientere und schnellere Entwicklung Wettbewerbsvorteile zu generieren.
Die Fallstudie SEAP bei Ericcson hat zwar signifikante Vorteile durch das agile Vorgehensmodell erkannt, jedoch unter der Prämisse, dass sich die Personalkosten bezüglich Sicherheit verfünffacht haben.
Dabei ist es fraglich, inwiefern hier tatsächlich eine Verbesserung bezüglich monetärer Kennzahlen erreicht wurde.
Für die Praxis lässt sich schlussfolgern, dass sicherheitskritische Bereiche durchaus auf agile Methoden setzen können, jedoch stets in angepasster Form.
Wichtig ist es dabei, die agilen Grundwerte nicht aus den Augen zu verlieren und die Gesamtsituation genau zu bewerten, um Verbesserungen oder Verschlechterungen zu erkennen.

Weitere Arbeiten in diesem Bereich sollten der Frage nachgehen, welche Möglichkeiten es gibt, die Auswirkungen von agilen Vorgehensmodellen in Kennzahlensystemen zu erfassen und zu bewerten.
Auch eine Durchführung von Fallstudien in weiteren sicherheitskritischen Bereichen wäre hilfreich. 
Hierbei sollte auf eine ausführliche Analyse unter Anwendung eines Kennzahlensystems gesetzt werden, damit klare Fakten bezüglich der Auswirkungen abgelesen werden können.
Des Weiteren sollten agile Vorgehensmodelle konkret an einzelne Normen angepasst werden, so dass fertige Vorgehensmodelle für die sicherheitskritischen Bereiche entstehen.

% DEVOPS AUSBLICK

