\chapter{Schluss} % 2 Seiten

Abschließend erfolgt eine kurze Zusammenfassung und ein Fazit sowie ein Ausblick, welche weiteren Arbeiten in diesem Bereich angestrebt werden könnten.

\section{Zusammenfassung}

Die vertikale Evolution beleuchtet die Ausweitung von agilen Methoden auf weitere nachgelagerte Schritte, wie Betrieb und Qualitätssicherung (DevOps).
Des Weiteren wird untersucht, ob DevOps auch in sicherheitskritischen Bereichen Anwendung findet beziehungsweise finden kann.
Der Einsatz von DevOps ermöglicht es, die Qualität eines Produkts deutlich zu steigern, die Time To Market und Time To Recover signifikant zu verringern und die Anzahl der Auslieferungen zu erhöhen, wie in der Fallstudie über die Einführung von DevOps bei Nordstrom erörtert wurde. 
Dies ermöglicht es Organisationen, sich einen großen Wettbewerbsvorteil zu erarbeiten, besonders in der Anfangszeit von DevOps. 
Für den Einsatz von DevOps spricht ebenso die einfache und flexible Einführung, da die Umstellung und Änderung einer bestehenden Organisation mit entsprechendem Aufwand und damit auch mit Kosten verbunden ist. 
So kann DevOps in Koexistenz zum \enquote{De-facto-Standard} ITIL eingesetzt werden, indem die Bereiche umgestellt werden, in denen DevOps die größten Vorteile bringt. 
Zudem ist der Umfang und die Qualität der Toolunterstützung für DevOps in den letzten Jahren stark angestiegen, wie in \autoref{sec:anwendung_devops} beschrieben.

Die Betrachtung der horizontalen Evolution beschäftigt sich mit der Frage, ob und wie agile Vorgehensmodelle und deren Methoden in sicherheitskritischen Bereichen angewendet werden können.

Damit die Fragestellung einen validen Ausgangspunkt hat, werden in den Grundlagen die klassischen und sicherheitskritischen Bereiche miteinander verglichen.
Hier wird deutlich, dass sicherheitskritische Bereiche einen stark erhöhten Sicherheitsanspruch an die Software haben, andere (strengere) Normen Anwendung finden, bei den Entwicklungspraktiken auf stark formalisierte Vorgehensmodelle gesetzt wird und die Time To Market im Vergleich unwichtiger ist.

Zur Klärung der horizontalen Evolution wird zuerst betrachtet, wie sich agile Vorgehensmodelle in klassischen Bereichen auswirken.
Dabei wird deutlich, dass die generelle Kommunikation zwar verbessert wird, dies jedoch auch eine deutlich höhere Komplexität mit sich bringt, insbesondere in verteilten Teams.
Des Weiteren steigt die Produktivität, sofern die Teams wenig Fluktuation aufweisen und das Management es versteht, Menschen und deren Interaktionen gut zu managen.
Dies führt zur nächsten Beobachtung, dass die Unternehmens- und Teamkultur einen maßgeblichen Anteil daran hat, ob agile Vorgehensmodelle erfolgreich eingesetzt werden können.
Weitere Vorteile sind eine verbesserte Kommunikation zwischen Team und Kunde, eine erhöhte Projektsichtbarkeit, schnellere Reaktion auf Änderungen, höhere Zufriedenheit der Teams sowie eine schnellere Time To Market.

Die Untersuchung, ob und wie sich agile Vorgehensmodelle in sicherheitskritischen Bereichen einführen lassen, wird anhand mehrerer Kernpunkte durchgeführt.
Es wird deutlich, dass sich Scrum in der Raumfahrtindustrie grundsätzlich einsetzen lässt, wenn Scrum in Kombination mit einem klassischen Vorgehensmodell in Form eines hybriden Vorgehensmodells eingesetzt und angepasst wird.
Dabei müssen einige Komponenten von Scrum weggelassen werden, während sich andere sehr gut eignen.
Extreme Programming müsste noch stärker angepasst werden.
SEAP stellt einen neuartigen Ansatz dar, wie agile Vorgehensmodelle durch sicherheitsrelevante Betrachtungen und Prozesse ergänzt werden können, ohne die Agilität zu verlieren.
Dabei werden sicherheitsbezogene Rollen eingeführt die in engem Austausch mit dem Entwicklerteam stehen.
Des Weiteren wird eine iterative Risikoanalyse integriert.
Die Einführung von SEAP hatte bei Ericcson deutliche Verbesserungen in qualitativer sowie quantitativer Form bezüglich der Risikoanalyse und -vermeidung zur Wirkung. 
Dabei stiegen jedoch zugleich auch die Personalkosten an.
Eine weitere geeignete Erweiterung ist die Einführung von Abuser Storys, die potentielle Angriffe und deren Schaden einer User Story zuordnen.
Dass die Normen in sicherheitskritischen Bereichen grundsätzlich auch mit agilen Vorgehensmodellen erfüllbar sind, wird durch Analyse der relevanten Normen klar.
Für die Norm der ECSS wird eine beispielhafte Implementierung bei der Swedish Space Company dargelegt.

\section{Fazit \& Ausblick}

Die Betrachtung der aktuellen Entwicklung von DevOps führt zu der Schlussfolgerung, dass DevOps eine sehr interessante Option für eine Vielzahl an Organisationen darstellt. 
Allerdings handelt es sich bei DevOps um eine noch recht junge Bewegung. 
So befinden sich einige Bereiche, wie beispielsweise der Bereich der Sicherheit, noch in der Entwicklung. 
Daher ist der Einsatz von DevOps in sicherheitskritischen Bereichen vorerst nicht, oder nur in geringem Maße, möglich. Zudem müssen die Art der Aufträge für DevOps geeignet sein. 
Falls ein dezidierter Betrieb explizit gefordert ist oder der Betrieb, beispielsweise auf Grund des Datenschutzes, beim Kunden stattfinden muss, kann DevOps nicht eingesetzt werden.

Die untersuchte Literatur bezüglich der horizontalen Evolution legt den Schluss nahe, dass agile Vorgehensmodelle und Methoden grundsätzlich in sicherheitskritischen Bereichen eingeführt werden können.
Dazu sind jedoch noch Anpassungen in Bezug auf die Dokumentation an den Modellen notwendig, was wiederum zum Teil im Widerspruch zu den agilen Grundwerten steht.
Diese Anpassungen sind kritisch zu sehen und sollten in der Praxis sehr behutsam eingeführt werden, da ansonsten eine Formalisierung der agilen Methoden zu befürchten ist, was kontraproduktiv wäre.
Denn vor dem Hintergrund der zunehmenden Privatisierung der Raumfahrt sind in diesem Bereich dringend agile Vorgehensmodelle nötig, um im Umfeld des erhöhten Wettbewerbsdruck durch eine effizientere und schnellere Entwicklung Wettbewerbsvorteile zu generieren.
Die Fallstudie SEAP bei Ericcson hat zwar signifikante Vorteile durch das agile Vorgehensmodell erkannt, jedoch unter der Prämisse, dass sich die Personalkosten bezüglich Sicherheit verfünffacht haben.
Dabei ist es fraglich, inwiefern hier tatsächlich eine Verbesserung bezüglich monetärer Kennzahlen erreicht wurde.
Für die Praxis lässt sich schlussfolgern, dass sicherheitskritische Bereiche durchaus auf agile Methoden setzen können, jedoch stets in angepasster Form.
Wichtig ist es dabei, die agilen Grundwerte nicht aus den Augen zu verlieren und die Gesamtsituation genau zu bewerten, um Verbesserungen oder Verschlechterungen zu erkennen.

Weitere Arbeiten in diesem Bereich sollten der Frage nachgehen, welche Möglichkeiten es gibt, die Auswirkungen von agilen Vorgehensmodellen in Kennzahlensystemen zu erfassen und zu bewerten.
Auch eine Durchführung von Fallstudien in weiteren sicherheitskritischen Bereichen wäre hilfreich. 
Hierbei sollte auf eine ausführliche Analyse unter Anwendung eines Kennzahlensystems gesetzt werden, damit klare Fakten bezüglich der Auswirkungen abgelesen werden können.
Des Weiteren sollten agile Vorgehensmodelle konkret an einzelne Normen angepasst werden, so dass fertige Vorgehensmodelle für die sicherheitskritischen Bereiche entstehen.

Um das Einsatzgebiet von DevOps noch weiter auszubauen, müssten Anpassungen und Weiterentwicklungen im Bereich der Integration von Sicherheit stattfinden, sodass DevOps, wie agile Vorgehensmodelle aktuell, auch in sicherheitskritischen Bereichen zum Einsatz kommen kann.
Hierzu sollte untersucht werden, welche Möglichkeiten es dazu gibt und wie sich diese in der Praxis auswirken.
Für das Jahr 2016 ist eine Ausbreitung von DevOps auf 25 Prozent der IT Unternehmen prognostiziert \parencite[vgl.][]{Gartner:2015}. 
Dies bietet noch großes Potential für Unternehmen, um sich durch DevOps einen Wettberwerbsvorteil zu sichern.

 